\section{Introduction}


\section{Software description}
The programming language we used is Python 3.6 since this language offers a wide range of matrix modifications and is easy to use. Furthermore, it is not required to install any extra libraries apart from numpy. 

\subsection{How to use}
In the file \textit{votingManipulation.py} you enter your preference matrix as a numpy array in line 86. The rows are the voters and the columns are the voters preference. In line 93 you can choose what voting scheme should be used for evaluating a winner. If multiple voting schemes are given, each one will be evaluated one after the other. The output is written to a txt file named after the scheme that was used. This is done due to the fact that if the output was printed on the terminal it could happen that one could not see the whole output due to character limitations. 

\section{Experiments}
How do the voting schemes compare to each other with respect to risk that strategic voting happens, are there any other conclusions that can be drawn from your experiments and results?).

\section{Findings}


\section{Things that need to be done}
Include analytic considerations on the difficulty and complexity of extending your TVA towards capturing: 1) voter collusion; 2) counter-strategic voting; 3) both voter collusion and counter-strategic voting; and 4) all of the above in the case when TVA does not have perfect information